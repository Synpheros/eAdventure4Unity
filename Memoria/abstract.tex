% +--------------------------------------------------------------------+
% | Copyright Page
% +--------------------------------------------------------------------+

\newpage

\thispagestyle{empty}

\begin{center}

{\bf \Huge Abstract}

  \end{center}
\vspace{1cm}

Educational games (aka serious games, SG) are a powerful educational content that are not extensively used in education yet. SG are costly to produce and they are very dependent of the technology changes in software or hardware. For example, many SG were produced in Adobe Flash or Java can not be run in some of the newer devices. One of the pioneer SG used in schools “Science Pirates: The Curse of Brownbeard” is currently not available because it is been adapted to new operating systems. Therefore we should simplify the full SG life cycle to make them a reliable educational content. Al the e-UCM research team we created a java based authoring tool called eAdventure (eA) and many SG in collaboration with many institutions. To deal with the previously identified problems and to simplify the creation and maintenance of SG reusing our previous experience and content we have created uAdventure (uA). uA is an SG editor built on top of Unity3D that allows for the creation of educational adventure games without requiring programming. uA has the same simplified graphical editor that eA and it is able to import previous games developed with eA. As uA is built on top of Unity3D it allows for a simple exportation of games for different platforms and make the created games more resilient to technological changes (as it is expected that Unity3D will cope with that complexity).
Along this document, it's explained the development of the interpreter of eAdventure games, as well as the integration with the editor developed by Piotr Marszal, in which some contributions are made, developing new editors. Also, for a work of innovation, and to support the games whose developers can't invest the time to transform them to uAdventure's new system, it has been developed an standalone emulartor, able to import an run games produced with eAdventure on any platform or operating system. Finally, to support and improve the student assesstment part, RAGE has been integrated in the project infraestructure, allowing access to Learning Analitycs tools.

\vspace{1cm}

% +--------------------------------------------------------------------+
% | On the line below, replace Fecha
% |
% +--------------------------------------------------------------------+

\begin{center}

{\bf \Large Keywords}

   \end{center}

   \vspace{0.5cm}
eAdventure, Authoring Tool, Unity3D, Unity, Interpreter, Emulator, e-Learning, Videogames, Serious Games, uAdventure, Life cyrcle, RAGE, Learning Analytics, xAPI.

