% +--------------------------------------------------------------------+
% | Copyright Page
% +--------------------------------------------------------------------+

\newpage

\thispagestyle{empty}

\begin{center}

{\bf \Huge Resumen en castellano}

  \end{center}
\vspace{1cm}

Debido a la llegada de nuevas tecnologías, algunas de las herramientas, y productos generados con dichas herramientas, de gran valor, se están quedando obsoletos por la falta de adaptación. Este es el caso de eAdventure, un motor de videojuegos educativos del género aventura gráfica, desarrollado en la Universidad Complutense de Madrid, y programado en Java. En este proyecto se desarrolla una herramienta capaz de interpretar un videojuego generado por eAdventure y ejecutarlo en el motor de videojuegos Unity. Dicho motor, caracterizado por la facilidad de generar aplicaciones multiplataforma, su sencillez, y sus licencias gratuitas, hacen que sea un apropiado nexo de unión entre eAdventure y las nuevas plataformas.

Este intérprete se ha desarrollado de forma colaborativa con el proyecto de Piotr Marzsal "NOMBRE DEL PROYECTO", que reconstruye el editor de eAdventure en Unity, realizando una integración de ambos proyectos, y participando en el desarrollo de algunas partes de su reconstrucción.

\vspace{1cm}

% +--------------------------------------------------------------------+
% | On the line below, repla	ce Fecha
% |
% +--------------------------------------------------------------------+

\begin{center}

{\bf \Large Palabras clave}

   \end{center}

   \vspace{0.5cm}
   
eAdventure, Unity, Intérprete, Emulador, e-Learning, Videojuegos, Serious Game
   


