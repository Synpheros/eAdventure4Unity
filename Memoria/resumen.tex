% +--------------------------------------------------------------------+
% | Copyright Page
% +--------------------------------------------------------------------+

\newpage

\thispagestyle{empty}

\begin{center}

{\bf \Huge Resumen}

  \end{center}
\vspace{1cm}

Los videojuegos educativos, también conocidos como juegos serios, son una herramienta educacional muy poderosa, cuya utilización no está muy extendida en la educación. Estos Serious Games son costosos de producir, y son muy dependientes de los cambios tecnológicos, tanto en el Software como en el Hardware. Por ejemplo, multitud de Serious Games estaban producidos en Adobe Flash o Java, y hoy en día no pueden ser ejecutados en algunos de los dispositivos más nuevos. Uno de los pioneros de los videojuegos serios "Science Pirates: The Curse of Brownbeard", actualmente no está disponible porque no ha sido adaptado a los nuevos sistemas operativos. Por lo tanto, el ciclo de vida de los juegos serios debe ser simplificado para hacerlos una herramienta de confianza. En el equipo de desarrollo e-UCM se ha creado una herramienta de autoría de juegos serios basada en Java llamada eAdventure, así como multitud de juegos serios en colaboración con multitud de instituciones. Para lidiar con los problemas anteriormente identificados, y simplificar el proceso de creación y mantenimiento de juegos serios, y reutilizando la experiencia previa, se ha creado uAdventure. Este proyecto es un editor e intérprete construido sobre Unity3D, que permite la creación de videojuegos educativos sin requisitos de conocimientos de programación. Como uAdventure está construido sobre Unity3D, permite la exportación de videojuegos, de forma sencilla para múltiples plataformas, y los hace más resistentes a los cambios tecnológicos.
A lo largo de esta memoria, se explica el proceso de generación del intérprete de videojuegos, así como la integración con el editor desarrollado por Piotr Marszal, en el que se realizan aportaciones, generando editores. Además, para realizar una labor de innovación, y dar soporte a los juegos cuyos desarrolladores no puedan invertir tiempo en transformar sus videojuegos al nuevo sistema de uAdventure, se ha desarrollado un emulador independiente capaz de importar y ejecutar juegos producidos con eAdventure en cualquier plataforma. Finalmente, para dar soporte y mejorar la parte de evaluación de los alumnos, se ha integrado RAGE en la infraestructura del proyecto, permitiendo el acceso a herramientas de Learning Analitics.

\vspace{1cm}

% +--------------------------------------------------------------------+
% | On the line below, repla	ce Fecha
% |
% +--------------------------------------------------------------------+

\begin{center}

{\bf \Large Palabras clave}

   \end{center}

   \vspace{0.5cm}
   
eAdventure, Unity3D, Unity, Intérprete, Emulador, e-Learning, Videojuegos, Serious Games, uAdventure, Life cyrcle.
   


