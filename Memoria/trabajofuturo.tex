\chapter{Trabajo Futuro}

Para mejorar el proyecto, se plantea una pequeña lista de trabajo futuro que garantizará que, si se realiza correctamente, se alcancen los objetivos planteados del proyecto.

Esta lista contiene:
\begin{itemize}
	\item Mejorar el acabado gráfico, dando soporte a todos los cursores, pues actualmente sólo se soportan el cursor normal, y el cursor de mano cuando existe un elemento interactuable bajo el cursor.
	
	\item Dar soporte a todas las transiciones de eAdventure, pues únicamente se ha dado soporte a la transición \textit{Fade} que conecta dos escenas mediante un efecto de degradado.
	
	\item Implementar todos los tipos de efectos disponibles, pues, aunque los efectos más utilizados tienen soporte, aún existen determinados efectos que su soporte es nulo o parcial.
	
	\item Mejorar el soporte a las transiciones, así como implementar la necesidad de que el jugador se encuentre cerca de un elemento de la escena para para poder interactúar con el.
	
	\item Implementar el inventario, pues el jugador no dispone de inventario de objetos.
	
	\item Añadir la posibilidad de cargar y guardar partida utilizando la clase \textit{GameState} y un elemento de serialización.
	
	\item Implementar una pantalla intermedia en el emulador para permitir al usuario ver detalles de los juegos intermedios, cargar y borrar partidas, así como la posibilidad de desinstalar un juego importado.
	
	\item Mejorar el soporte con RAGE, añadiendo procesos de generación automática de cálculo de progreso y evaluación, pudiendo generar el primero a partir de la sucesión de escenas que conectan la escena inicial y la escena final, o el segundo, a partir de unas variables que determine el usuario.
	
	\item Integrar RAGE en el editor de forma transparente, permitiendo seleccionar desde los menús, que cosas se desean monitorizar, realizando un par de pulsaciones en botones.
	
	\item Mejorar la documentación dentro del código, pues el código en si mismo no tiene demasiada documentación, y pese a los diagramas de clases generados, una documentación de calidad incrementa la calidad del software.
	
	\item Rediseñar el editor generado por Piotr para integrarlo dentro de Unity. Pese a que la interfaz actual permite al usuario una interacción muy similar a la que tenía trabajando con eAdventure, dicha interfaz es muy antigua y puede ser mejorada mediante la integración completa en Unity, creando un editor híbrido.
	
	\item Dar soporte a cualquier objeto prefabricado, pudiendo utilizar elementos tridimensionales dentro de la escena.
\end{itemize}

Sin embargo, este listado de trabajo futuro no es más que un listado de características que mejorarían la experiencia de usuario de uAdventure, por lo que, cualquier característica adicional que se le desee incorporar al editor que mejore esta experiencia puede considerarse como trabajo futuro del proyecto. Conocer este listado de características será determinado por el uso que los usuarios den al proyecto, mediante la obtención de retroalimentación por su parte.