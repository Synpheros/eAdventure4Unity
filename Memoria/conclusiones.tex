\chapter{Conclusiones}
\label{conclusiones}

Tras haber completado el proyecto, incluyendo el diseño, la implementación y la escritura de esta memoria, se extraen determinadas conclusiones que se deben abordar en este \textit{post mortem} del proyecto. Primero se realizará una reflexión personal acerca de los resultados del proyecto y luego se presentarán conclusiones categorizadas dentro de los diferentes objetivos que se plantearon, estableciendo una relación entre objetivos planteados y resultados obtenidos.

Bajo mi impresión personal, estoy bastante satisfecho con el resultado obtenido, pues, en mayor o menor medida he completado la lista de objetivos que me planteé, aprendiendo gran cantidad de nuevos conceptos. Mediante la experimentación he aprendido que, al menos en mi caso, trabajo mejor realizando un pequeño diseño, en papel, de cómo me imagino que deberían funcionar las cosas, y desarrollando un prototipo que realice lo implementado, y me permita ver si el resultado obtenido es el esperado, o se necesita realizar un rediseño de la aplicación.

Una de las conclusiones que extraigo del proyecto es que es muy satisfactorio trabajar con un estudiante extranjero, de una cultura diferente, como es Piotr Marszal. La experiencia de trabajar con una persona que se toma su trabajo tan en serio, al que le puedes pedir resultados y responde ante tus necesidades, es una experiencia muy buena, pues te permite crecer como persona al conocer otra forma de trabajar, y hace que te impliques más en el desarrollo del proyecto, desarrollando un código de mayor calidad.

El listado de objetivos, planteado en el capítulo \ref{objetivos}, de forma simplificada, es el siguiente:

\begin{enumerate}
	\item Hacer funcionar el videojuego Checklist en Unity3D.
	
	\item Hacer funcionar el mayor número de juegos desarrollados por CATEDU.
	
	\item Hacer funcionar cualquier juego de eAdventure en Unity3D.
	
	\item Generar un diseño y arquitectura de aplicación de calidad.
	
	\item Integrar este intérprete con el proyecto de Piotr Marszal.
	
	\item Implementar una serie de facilidades que mejoren la interacción en el ámbito táctil.
	
	\item Dar la capacidad a los no desarrolladores, de poder jugar a videojuegos generados con eAdventure en cualquier plataforma, y sin necesidad de tener Unity, así como de beneficiarse de las características de uAdventure.
	
	\item Mejorar la capacidad de evaluar existente en eAdventure.
\end{enumerate}

\textbf{Acerca del objetivo 1}: Este objetivo fue uno de los primeros en completarse, pues tras la construcción del prototipo inicial, el videojuego Checklist, descrito en el apartado \ref{checklist}, ya era completamente jugable. Pese a que el resultado obtenido en dicho momento del desarrollo no era, a nivel de arquitectura de proyecto y de representación visual del mismo, un resultado tan bueno como para poder producir dicho juego como juego final, según el proyecto avanzó, se implementaron una serie de mejoras que consiguieron que el juego Checklist pudiese generarse y ser un videojuego de calidad.

Para satisfacer este objetivo fue necesaria la implementación del prototipo inicial del proyecto al completo, presentado en el apartado \ref{primeraiteracion}. Realizando una investigación acerca de los paquetes ".jar" ejecutables de los juegos, explicada en el apartado \ref{comenzando}. Tras esta investigación, se desarrolló una primera arquitectura del proyecto, y se implementaron las clases necesarias para poder leer el fichero de especificación del juego, y almacenar los datos que se presentaban en dicho fichero. A estas clases se las conocería como Clases de Datos. Tras esto se generaron clases capaces de interpretar dichas Clases de Datos y representarse en el videojuego, conocidas como Clases de Comportamiento. Finalmente, y debido a la necesidad de funcionalidad adicional, se implementaron clases que daban soporte a efectos y a conversaciones, así como una clase controladora del juego, llamada \textit{Game}, capaz de gestionar la interacción y el estado del juego.

\textbf{Acerca del objetivo 2}: Los juegos de CATEDU son una gran cantidad de juegos, entre los cuales se encuentran el juego de Primeros Auxilios, o el juego de Viaje a Londres. Para dar soporte a este objetivo fue necesaria la realización de la segunda iteración del proyecto, aunque no en su totalidad. En esta segunda iteración se añadieron nuevas características no soportadas en la primera, como la gestión de Temporizadores, explicada en el apartado \ref{timercontrollersecit2}, la capacidad de mover objetos a lo largo de la pantalla, explicada en el apartado \ref{itemsit2}, la posibilidad de que hubiera juegos en tercera persona, explicadas en el apartado \ref{playerit2}, y la capacidad de que, en estos, el jugador pudiera moverse, explicados en el apartado \ref{trajectoryit2}.

Finalmente, para soportar totalmente estos videojuegos fue necesario dar soporte a contenido multimedia audiovisual, como son los videos y el audio. El soporte a estos elementos está explicado en los apartados \ref{resourcemanager} y \ref{scene}.

\textbf{Acerca del objetivo 3}: Dado que conseguir que literalmente cualquier juego de eAdventure funcione en uAdventure es una tarea compleja, pues requiere de implementar todas y cada una de las características de eAdventure, por el momento y muy lamentablemente, es posible que algunos de los juegos que se intenten hacer funcionar, puedan producir algún fallo por la falta de implementación de alguna de las características de eAdventure. Sin embargo, y para intentar garantizar esta característica, se han implementado prácticamente todos los efectos para las macros, efectos y acciones, explicado en el apartado \ref{sequencessecit2}.

\textbf{Acerca del objetivo 4}: Para satisfacer este objetivo ha sido necesaria la realización de multitud de iteraciones sobre el código del intérprete de uAdventure, mediante la refactorización y el continuo diseño de elementos de la aplicación se ha conseguido alcanzar un diseño que se apoya sobre los pilares de la Ingeniería del Software, y senta sus bases para ser lo más extensible posible, replicando la menor cantidad de código posible, basándose en patrones como el patrón \textit{Singleton}, o \textit{Factory}, implementando Gestores y Controladores para la gestión y control de tareas, etc... Toda esta labor de diseño además, está acompañada de diagramas de clase que presentan la arquitectura, diseño de clases e implementación a lo largo de la segunda iteración del proyecto, presentada a lo largo de todo el capítulo \ref{it2} de esta memoria.

\textbf{Acerca del objetivo 5}: Para la integración de los proyectos ha sido necesario generar un modelo de datos común para las dos aplicaciones, para que ambos proyectos se construyeran utilizando la misma base. Sin embargo, no sólo el modelo de datos se ha desarrollado, sino que también se desarrolló en conjunto un lector de ficheros de especificación de juegos de eAdventure. Todo esto está explicado en el apartado \ref{coreit2}. Por otra parte, en el proceso de integración, en este proyecto se generaron dos editores para el editor de uAdventure desarrollado por Piotr. Dichos editores están explicados en el apartado \ref{sequencesit3}.

\textbf{Acerca del objetivo 6}: Para mejorar la interacción en el ámbito táctil se da soporte en el proyecto al uso de dos \textit{Shaders}, los cuales permiten hacer que los elementos interactuables de la escena brillen y se hagan visibles para el usuario que no dispone de un cursor de ordenador para desplazarlo a lo largo de la escena para identificar los elementos interactuables. Por otra parte, se recoloca la interfaz de los juegos en algunos determinados momentos, como en las listas de respuestas, donde las respuestas ya no aparecen como pequeñas líneas de texto en la parte inferior de la pantalla, sino que se presentan como un listado de grandes botones cendrados en la pantalla. Estas características se explican en el apartado \ref{apearanceseccionit2}, así como en el apartado \ref{guimanagersectionit2}.

\textbf{Acerca del objetivo 7}: Para dar soporte a los no desarrolladores, para que puedan beneficiarse de las ventajas de uAdventure, y poder utilizar cualquier juego de eAdventure en cualquier plataforma, se ha desarrollado un emulador independiente de videojuegos de eAdventure, capaz de importar juegos en formato ".jar", y permitir al usuario jugarlos. Este emulador está explicado en el apartado \ref{emulatorit3}. Este emulador dispone de distintas vistas, entre las que se presenta un explorador de archivos capaz de explorar el sistema de ficheros para permitir al usuario seleccionar el videojuego que desee importar, así como una pantalla principal donde se presentan todos los videojuegos importados, o una pantalla de configuración.

\textbf{Acerca del objetivo 8}: para mejorar la capacidad de evaluar existente en eAdventure, se ha conectado el sistema con RAGE, el cual provee, mediante una interfaz web, una serie de páginas con información importante y gráficas acerca del progreso y evaluación de los alumnos de forma en grupo e individual, permitiendo al profesor enfocar los esfuerzos en aquellos alumnos que hayan quedado rezagados o que demuestren problemas en su aprendizaje. Para ello se ha dado soporte a un proyecto del grupo e-UCM, que consiste en generar un \textit{Tracker} que se comunique con RAGE. Dicha comunicación está explicada en el apartado \ref{ragetrackerit2}. Por otra parte, y para generar los perfiles de evaluación, se ha facilitado el acceso a RAGE desde eAdventure, generando un editor y una serie de clases comunes con RAGE, capaces de almacenar elementos de este sistema. Dicho editor se explica en el apartado \ref{rageeditorit3}.

\chapter{Conclusions}

After completing this project, including the design, implementation and writing of this report, we have identified certain conclusions that are being addressed in the post mortem of this project.certain conclusions to be addressed in this post mortem of the project are extracted. They will be presented categorized within the different objectives that we identified, establishing a relationship between the objectives and the results.

Under my personal impression, I am quite satisfied with the result, therefore, in greater or lesser quantity, I have met the target list that i wrote, learning many new concepts. Through experimentation I have learned that, at least in my case, I produce better projects making a small design on paper, how I imagine things should work, and developing a prototype that implements that design, and lets me see if the result obtained it is the expected, or need to make a redesign of the application.

One conclusion I get from the project is that is very satisfying to work with a foreign student, a different culture, such as Piotr Marszal. The experience of working with a person who takes his job so seriously, which you can order results and responds to your needs, is a very good experience, because you can grow as a person to know another way of working, and makes you involve yourself in the development of the project, developing a code of higher quality.

The simplified list of objectives identified in Chapter 2 is the following:
1. Make the Checklist videogame work in Unity3D .
2. Make the highest number of games developed by CATEDU work in Unity 3D.
3. Correctly run any eAdventure game in Unity3D .
4. Design a quality application architecture.
5. Integrate the implemented platform with Piotr Marszal project.
6. Implement a set of interaction improvements for mobile devices.
7. Develop a functionality to allow non-developers to be able to play games generated with eAdventure on any platform, without having Unity installed, as well as
benefit from the characteristics of uAdventure.
8. Improve the existing eAdventure assessment functionality.

\textbf{About the first goal}: This goal was one of the first to be completed, since
after the construction of the initial prototype, the game Checklist, described in section \ref{checklist} was fully playable. Although the results obtained at that time
were not representative of a high quality videogame. The quality of the Checklist videogame improved significantly after several iterations.


To satisfy this objective was necessary to implement the initial prototype of
whole project, presented in chapter \ref{primeraiteracion}. Researching about “.jar”
packages and runnable games, explained in Section 3. After this research,
we developed the first project architecture, and we implemented the necessary classes read and store the data specified inside the game’s specification file. These classes are known as the Data Classes. Afterwards, we generated the Behavior Classes (classes that interpret and run the Data Classes). Finally, we implemented the classes required for additional functionalities such as support for effects, conversations and a controller known as Game, able to manage the game state and operations over its data model.



\textbf{About Objective 2} : CATEDU is composed of a set of games such as the First Aid Game, or the game Travel to London. We had to iterate a second time over the project, although not entirely. In this second iteration new features were added such as Timer management , explained in section \ref{timercontrollersecit2}, the ability to move objects across the screen, explained in section 5.2.3, the support for third person games, explained in section \ref{itemsit2} , and the player movement in these types of games, explained in sections \ref{playerit2}, and \ref{trajectoryit2}.

Finally, for give full suport to this videogames, it was necesary to give support to multimedia content, like videos and audio. Support to multimedia is described in sections \ref{resourcemanager} and \ref{scene}.

About Objective 3 : Correctly running every eAdventure game in uAdventure is a very complex task because it requires the implementation of all the original eAdventure functionalities. Currently, it is possible that some of the existing games may not be fully runnable on uAdventure. However, to maximize the amount of functionality implemented, we fully support macros, effects and actions, explained in section \ref{sequencessecit2}.

\textbf{About Objective 4}. To complete this objective we had to iterate several times over the uAdventure source code. We accomplished a source code that is flexible and implements several Software Engeneering design patterns such as Singleton or Factory (implementing Managers and Controllers for the operations process). This refactoring and redesign process is  supported by class diagrams for the architecture and the class implementation documented in chapter 5 \ref{it2}.

\textbf{About Objective 5}. To be able to integrate both projects it has been necessary to develop a common data model. We have developed the common data model and a file reader designed for the original eAdventure files and explained in section \ref{coreit2}. During the integration process, we also ccreated two editors that have been explained in section \ref{sequencesit3}.

\textbf{About Objective 6}. To improve the interaction functionality, especially designed for mobile devices, we have developed two different shaders. The shaders highlight elements of interest of the scene. This is useful when there is no cursor available on the screen. We also redesigned parts of the user interface to improve it for mobile devices. All these improvements are described in section \ref{guimanagersectionit2}.

\textbf{About Objective 7} In order to allow users with no programming skills use the uAdventure framework and be able to run eAdventure games on different platforms we have developed an emulator able to import and play games formatted as “.jar” packages. The emulator consists of a file explorer that allows the user to select the games to be imported. This is fully explained in section \ref{ragetrackerit2}.

\textbf{About Objective 8} To improve the current assessment functionality and the evaluation support of eAdventure, we have connected the system with RAGE. RAGE provides the analysis infrastructure required to process the progress and other related learning outcomes. We developed the functionality required to manage the data model of RAGE (such as games, classes and sessions) and we used a tracker asset developed by the e-UCM research group to communicate with the RAGE platform, explained in section \ref{rageeditorit3} 



